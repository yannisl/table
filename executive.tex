% Copyright 2017 Y Lazarides
% 
% Licensed under the Apache License, Version 2.0 (the "License");
% you may not use this file except in compliance with the License.
% You may obtain a copy of the License at
% 
%      http://www.apache.org/licenses/LICENSE-2.0
% 
% Unless required by applicable law or agreed to in writing, software
% distributed under the License is distributed on an "AS IS" BASIS,
% WITHOUT WARRANTIES OR CONDITIONS OF ANY KIND, either express or implied.
% See the License for the specific language governing permissions and
% limitations under the License.

\chapter{Executive Summary}

\epigraph{\textit{Question}: How do Projects get delayed by two years?\\
                \textit{Answer}: \ldots One day at a time.}

\noindent This report is a high level report examining the status of all aspects of the Project, the successes during the last month, 
current constraints, execution risks and provide recommendations for Management action.

\section*{Change of Project Director}

My arrivng onto the project has been received positively by all staff, Client, AECOM and REDCO. I had numerous meetings with all of them. James Lawson has adjusted to the role of Deputy Project Director and so far has been contributing positively in reporting, \hl{planning and other tasks}. 

\section*{Initial Focus}

My initial focus for the first 30 days was to make better arrangements for all departments, increase the productivity and motivation of all staff and focus on problematic areas. We also managed to remove  some long standing items from minutes. The problematic areas were identified, as Engineering, Construction, Materials, Procurement, Commercial,  Accounts, QA/QC, Safety (on Redco's side), lack of any Document Control system. No clear cut roles were defined or targets set and most departments have been badly staffed. 

\section*{Major Milestones}

One problem with the current predicament is the lack of any commitment from all parties, or understanding of the milestones (at all levels) and program end dates, which I list below:
\medskip

\begin{tabular}{ll}
Wild Air (Redco Rev-15) 	                & 01-May-17	\\
Power ON (Redco Rev-15)						 & 30-Sep-17	\\									
MEP Project Completion			             & 06-Mar-18	\\									
Main Works Completion	    	            & 31-May-18	\\									
\end{tabular}

The Wild Air milestone, is a signal that by that time all installations should be complete,  as it enables Finishing Subcontractors to install sensitive finishes, while five months later the Power On can be translated as the start of Commissioning. Despite the highly visible progress of the Main Contractor in casting, in reality they are behind schedule and we could have recorded delays.  However without our own Programme ready and on the receiving end in terms of Material  Submittals the Main Contractor can easily claim delays against us, as soon as he removes the props in areas casted and opens construction to us.

In terms of turnover we would need to average in excess of 70 million QAR a month. If we are to achieve it,  decisive and parallel action is necessary from all including the acceptance of risk. The danger of delays from our end is that we will experience the ``mother of all cash-flow problems'' down the line, as large turnovers and the Contract structure of Material on Site payments can cause such cash-flow bottlenecks, even if the Contract has adequate mark-ups. 

\section*{Engineering}

Bottlenecks of information were removed, by meeting with Shaker daily and we expect that by the end of March all IFC drawings and a revised report will be made available. 

Drawing production has improved, drawing registers are now being kept more accurately and  daily production reported.  There are a number of  reasons for the lower production, mainly lack of expertise in the drawing office, a slack attitude and bad communication with Virtual and Vipin in Egypt and India respectively.  We have had two meetings with Vipin, and he has already send us one-coordinator and one is awaiting visa. We have also made arrangements for four additional CAD Operators to join us, once visas are ready. 

My expectations are that we should be able very soon to be ahead of the backlog (already AECOM complaining that they receiving too many drawings). However, I have doubts as to the quality of the output.

Current staff in the drawing office are lacking in many respects and there is no head mechanical or electrical. Savva Nicolaides does not make the grade and should be removed. Akheel is highly experienced but does not have a production mentality and a waste of money to be developing builder's works drawings. He should be in a position to work on the difficult aspects of the works (power related, LV rooms, panels, busbars and the like) and to then move onto construction.

For part solutions please see the section under Construction.

Garry Worthington is the type of Manager that could run the Engineering Department, provided everything was in place. I would recommend that we find two strong Engineers to lead the Mechanical and Electrical sections and to have as standby should there be a need. These two Engineers are necessary irrespective of the alternate solution offered later on under the Cosntruction section.



\section*{Construction}

Construction has so far kept pace with slab castings, but with the Main Contractor having now introduced 24hr shifts is badly understaffed. Current staff are mostly inexperienced for fast-track large development work. 

The staffing levels are inadequate and has also impacted onto MTO. We have started with the construction of both the Site Offices, as well as the Site Stores and this is expected to be completed by end of March.

Material take-offs, ordering and related activities should be carried out by Material Engineers (not on the chart and something I would like to introduce). They need to have an MEP background and be in a position to work in a highly pressurized environment. 


\begin{longtable}{lll lc}
\toprule
\rowcolor{thetableheadbgcolor}s/n   &  Job Title                                                      & Name           &Required By    &  Duration\\
\midrule
1     &   Mechanical Engineer                                    & Vacant         &immediate       & end of project  \\
2     &   Mechanical Engineer                                    & vacant         & immediate       &17  \\ 
3     &  Electrical Engineer                                        & Vacant         & immediate      &end of project  \\ 
4     &  Electrical Engineer                                        & Vacant         & immediate      &17 \\
\bottomrule
\end{longtable}

The above engineers can also be utilized for specific Material Submittal issues, as well as handle time consuming tasks, such as checking thoroughly through detailed Technical submittals (fans, fcus, DBs etc) to decouple this activity from Engineering to let them focus on drawing production. Hence at least two of them should have a good command of English and the ability to handle detailed discussions with Suppliers and AECOM.

Construction Managers should be deployed as per the below dates and rather leave earlier if there is a budget issue. 

\begin{longtable}{lll ll}
\toprule
s/n   &  Job Title                                                      & Name                       &Required By    &  Duration\\
\midrule
1     &   Construction Director                                   & Costas Assiotis         &    & Full  \\
2     &   Construction Managers (Slabs/Shafts)           & Terrence O'Connor    &    & end of activity  \\ 
3     &  Construction Manager (Car Park)                   & Vacant                      & April 2016   & end of activity \\ 
4     &  Construction Manager (Retail)                       & Vacant                      & May 2016    & end of activity\\
5     &  Construction Manager (Residential)               & Vacant                      & May 2016   & end of activity\\
6     &  Construction Manager (Hotel)                       & Vacant                      &  June 2016   & end of activity\\
\bottomrule
\end{longtable}

We need to bring forward all Senior Electrical and Mechanical Engineers that are on the Organization Chart for Construction and let them contribute to both Site and Engineering at this stage. There are 10 on the chart and we have so far only one on site. The Theme Park ones need to also come on board as we have received highly
detailed drawings and we need to be in a position to start developing drawings soon.

\begin{longtable}{lll ll}
\toprule
s/n   &  Job Title                                                      & Name                       &Required By    &  Duration\\
\midrule
1     &  Senior Electrical                          & Vacant        			 & immediate    & Full  \\
2     &  Senior Electrical                          & vacant    			&  30 April  & end of activity  \\ 
3     &  Senior Electrical                          & Vacant                      &  immediate   & end of activity \\ 
4     &  Senior Electrical                          & Vacant                      & 30 April   & end of activity\\
5     &  Senior Elecrical                           & Vacant                      &  30 April  & end of activity\\
6     &  Senior Mechanical                       & Vacant                      &  immediate  & end of activity\\
7     &  Senior Mechanical                       & Vacant                      &   immediate & end of activity\\
8     &  Senior Mechanical                       & Vacant                      &  immediate  & end of activity\\
9     &  Senior Mechanical                       & Vacant                      &  30 April  & end of activity\\

\bottomrule
\end{longtable}

A similar approach should be taken for Site Engineers to match the deployment of the Seniors.

\begin{longtable}{lll ll}
\toprule
s/n   &  Job Title                                                      & Name                       &Required By    &  Duration\\
\midrule
1     &  Site Electrical                          & Vacant        		   & immediate    & Full  \\
2     &  Site Electrical                          & vacant    			   &  30 April  & end of activity  \\ 
3     &  Site Electrical                          & Vacant                      &  immediate   & end of activity \\ 
4     &  Site Electrical                          & Vacant                      & 30 April   & end of activity\\
5     &  Site Elecrical                           & Vacant                      &  30 April  & end of activity\\
6     &  Site Mechanical                       & Vacant                      &  immediate  & end of activity\\
7     &  Site Mechanical                       & Vacant                      &   immediate & end of activity\\
8     &  Site Mechanical                       & Vacant                      &  immediate  & end of activity\\
9     &  Site Mechanical                       & Vacant                      &  30 April  & end of activity\\

\bottomrule
\end{longtable}
\section*{Procurement}

Procurement has picked up pace, but in my opinion we need to give Marios more supporting staff in order for him to be able to concentrate in negotiations and the identification of resolution of problematic areas. We have agreed a target of closing one material a day, which is the rate we need to keep-up with the Project requirements. 

In order to keep control both with paper work, as well as actual materials, he will need a strong person to assist him with this to the point that he can fully unload these tasks. He will also without saying need a Central Store. A decision on renting should be taken immediately and the Stores and Systems be in place latest by end April early May. 

\section*{Commercial}

Stephen is a Quantity Surveyor and not a Commercial Manager. He has the ability to cope, but needs the supporting staff urgently. We have been unable so far to bring over the 1 QS which has been employed. His full compliment needs to be on site as soon as possible. I would also like to see a strong Number 2, in case we have a resignation down the line or a Commercial Manager as his Senior. 

\section*{Administration and Human Resources}

Site Administration so far has been coping and supporting the Site well. I have a concern regarding the issuing of visas which is a bit slow. There need to be closer co-operation between Dubai recruitment and Doha Administration regarding this aspect. Currently we have 250 visas available for staff and 563 for tradesmen. The latter is problematic as it includes 218 Bangladeshi and 100 Tunisians which are pretty much unusable for MEP and we might have problems converting to other trades. The current visas will have to be used before we can re-apply for new ones (I am pretty much sure that we can get more than 2000 on the basis of the current contract, and the opportunity should not be lost). 

I have a concern also as to helpers. My recommendation is that the Company does not hire helpers. It is preferable to hire at a salary level of just above helper, tradesmen that did not make the grade, e.g. a Plumber that has failed the trade test can be made an offer of Plumbing Assistant. This can provide the Company with a resource to train down the line, an avenue and an incentive for the person to grow with the Company as well as a virtually unlimited supply of skills. 

\section*{QA/QC}

The QA/QC Department has been battling to keep up with routine WIRs. We average about 7-9 a day. I think the problem is at the top and request replacement of the QA/QC Manager, as well as the adition of another 2 QA/QC personnel (experienced so they can train the trainees). 

\section*{Safety Department}

This is the only department so far that is working satisfactorily, despite shortages of personnel. The reasons as to why this Department is functioning properly should be food for thought for all of us. 

\section*{Planning Department}

I have no doubt that Lakew has the skills to manage the department. However, we have still not managed to issue a workable programme. The three assistants he has, are inexperienced and none of them have the personality or the level of MEP understanding to handle meetings with Client or Main Contractor.

I believe we have put all our eggs in one basket in this respect and recommend the employment of a second in command with the potential to become a Planning Manager down the line. The Client has two planners at the level of Lakew and Redco has a strong internal Planning department, as well as an outside party assisting them.



\section*{Accounts and Finance}

No systems are currently in place to have a good up-to date Management Information System regarding the financials of the Project. We had a meeting with J\&P related personnel as to how to improve the flow of paper-work with some promising results but it is clear that information will trickle back to us in a limited form.  

Most of my discussions today with our Group Financial Manager revolved around ``Conspiracy Theories''. No doubt ``matzaranges'' happen at all levels, but since we are controlling the flow of documents to J\&P we should be in a position to catch any problems in this respect from our side via a rigid and definable set of procedures and regular audits.  Incidentally it was clear in our meeting with J\&P that the same ``Conspiracy Level'' attitude prevails from their end (with us being the subject). 

As long as these sort of issues do not interfere with the flow of funds to the Project and the timely issuing of payments where necessary, it is all very amusing and a challenge to higher Management to resolve. 

\section*{MEP Delivery Strategy}

I am currently developing the strategy and hopefully will have it ready in a few days, although a preliminary version will be ready by Tuesday. I have focused the strategy on accelerating Engineering and start-up Construction activities at the beginning (to reduce requirement for tradesmen down the line), securing large sections of the works such as Apartments early and introducing a production orientated site organization and pipeline. 

\section*{Risks}

The biggestt financial risk at present is that we might not be able to spent the Provisional Sum for the Theme Park, as a number of items were included with the original tender. This might be true to some extend with the rest of the Provisional Sums.

Time related risks at the moment are abound in Engineering, Procurement and especially Construction.

\section*{Conclusions and Recommendations}

Currently we running a high risk to achieve the programme target dates, as well as to keep our good relations with Main Contractor and Client. Staffing levels for Engineering, Construction, Procurement and Commercial need to be increased within the next 20 days if we are to achieve a turn-around.




\endinput

One of the largest
obstacles in detailing MEP services for hotels, is the integration and co-ordination of the MEP requirements with the requirements dictated by the Interior Design. Both RCP and ID layouts at the Shop Drawing level need to be available to detail final MEP requirements. 

Our original strategy was based on initially focussing at the guestroom floors for all three hotels, where we expected that it would have been easier to provide ID and RCP layouts. As the balance of the information was made available we could then plan to execute the works in the public areas, such as ballrooms, restaurants and the like. Another area that offered early access was the car parking areas in the basements. Shop drawings were completed early and works in this area are approximately 85\% completed. Remaining works in Car Parks relate to fans (still to be finalized) and cabling for same.

Between these two broad categories: guestroom floors and car parks,  we can open approximately 70\% of the works by area, allowing  time for designs to settle for the rest of the buildings (Podia). Currently access to guest-room floors is only available in Westin and St Regis. The "W" Hotel has been seriously delayed due to Civil works issues.

The report does not touch on all the aspects of the MEP works, as we intended it to be short and focused. We hope that it will elicit both internal and external dialogues that will lead to refinements and improvements. 
\bigskip

{\tt

 \hbox to \textwidth {\hfill Dr.~Y Lazarides\hspace{1em}}

 \hbox to \textwidth{\hfill Project Director\hspace{1em}}

\hbox to \textwidth{\hfill HLS-DSE JV\hspace{1em}}
}
